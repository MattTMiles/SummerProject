\documentclass[a4paper,fleqn,usenatbib]{mnras}

\usepackage{newtxtext,newtxmath}
% Depending on your LaTeX fonts installation, you might get better results with one of these:
%\usepackage{mathptmx}
%\usepackage{txfonts}

\usepackage[T1]{fontenc}
\usepackage{ae,aecompl}

\usepackage{numprint}
\usepackage{booktabs}
\usepackage{float}

\usepackage{amsbsy}
\usepackage{graphicx}	% Including figure files
\usepackage{color}
\usepackage{amsmath}	% Advanced maths commands
\usepackage{amssymb}	% Extra maths symbols

\newcommand{\todo}[1]{\textcolor{red}{#1}}

%%%%%%%%%%%%%%%%%%%%%%%%%%%%%%%%%%%%%%%%%%%%%%%%%%

%%%%% AUTHORS - PLACE YOUR OWN COMMANDS HERE %%%%%

% Please keep new commands to a minimum, and use \newcommand not \def to avoid
% overwriting existing commands. Example:
%\newcommand{\pcm}{\,cm$^{-2}$}	% per cm-squared

%%%%%%%%%%%%%%%%%%%%%%%%%%%%%%%%%%%%%%%%%%%%%%%%%%


% Title of the paper, and the short title which is used in the headers.
% Keep the title short and informative.
<<<<<<< HEAD
\title[On the discovery and origin of r-process stars in the Milky Way.]{On the discovery and origin of $r$-process stars in the Milky Way.}
=======
\title[Discovery and origin of $r$-process stars]{On the discovery and origin of $r$-process stars in the Milky Way}
>>>>>>> 71dfab2a8ca9822ca18c45140f7371a65edee904

% The list of authors, and the short list which is used in the headers.
% If you need two or more lines of authors, add an extra line using \newauthor
\author[Matthew T. Miles et al.]{Matthew T. Miles,$^{1}$\thanks{E-mail: mtmil3@student.monash.edu}
	Andrew R. Casey,$^{1,2}$
	Brodie J. Norfolk,$^{1}$
	Alex J. Kemp,$^{1}$\newauthor
	John C. Lattanzio,$^{1}$
	Kevin C. Schlaufman,$^{3}$
	Alexander P. Ji,$^{4}$
	Anna Y. Q. Ho$^{5}$
	\\
	% List of institutions
	$^{1}$School of Physics and Astronomy, Monash University, Clayton Campus, Victoria 3800, Australia\\
	$^{2}$Faculty of Information Technology, Monash University, Clayton Campus, Victoria 3800, Australia\\
	$^{3}$Department of Physics and Astronomy, Johns Hopkins University, 3400 N Charles St., Baltimore, MD 21218, USA\\
	$^{4}$Department of Physics and Kavli Institute for Astrophysics and Space Research, Massachusetts Institute of Technology, Cambridge, MA 02139, USA\\
	$^{5}$Cahill Center for Astrophysics, California Institute of Technology, MC 249-17, 1200 E California Blvd, Pasadena, CA, 91125, USA
}

\date{Accepted XXX. Received YYY; in original form ZZZ}

\pubyear{2018}

\begin{document}
	\label{firstpage}
	\pagerange{\pageref{firstpage}--\pageref{lastpage}}
	\maketitle
	
	\begin{abstract}
		The production of heavy elements occurs through the slow and rapid neutron capture process. Numerous sites satisfy the conditions needed for the r-process to occur, but the relative frequency of these sites is unknown largely due to the lack of a large number of stars with strong r-process signatures. Here we report the discovery of 47 r-process enhanced stars discovered in the LAMOST survey, more than doubling the known number in our Galaxy. To our knowledge, we provide the first estimate on whether neutron star mergers are frequent enough to account for all known r-process stars in the Milky Way. \todo{We conclude....}. \todo{The discovery of these stars implies multiple r-process sites.}
		% contrary to previous conclusions in the literature including all r-process material being produced in small amounts in common events such as core-collapse supernovae, or the material exclusively being produced in very large amounts in very rare events, such as neutron star mergers. 
	\end{abstract}
	
	\begin{keywords}
		nucleosynthesis -- r-process -- neutron star mergers -- core-collapse supernovae
	\end{keywords}
	
	
	\section{Introduction}
	
	Heavy elements ($Z > 30$) are synthesised through the slow (s) and rapid (r) neutron capture processes \citep{Sneden2008}. The nucleosynthesis of these elements has been a key area in astrophysics research since it was first theorised  \citep{Burbidge1957}.
	
	Potential sites of the r-process have been discussed for over sixty years. Rapid neutron capture occurs when the neutron density is high enough that the rate of neutron capture will occur faster than the rate at which the isotope undergoes a $\beta$-decay. The rapid neutron capture sites are very specific, with the most favoured being core-collapse supernovae, neutron star mergers, or magnetorotationally driven supernovae. Core-collapse supernovae are relatively common throughout the Galaxy, but they only produce small amounts of r-process material. Neutron star mergers or magnetorotationally driven supernovae, however, occur much less frequently than core-collapse supernovae, but produce far more r-process material. Events such as these are theorised to account for the case of the highly r-process enhanced ultra-faint dwarf (UFD) galaxy Reticulum II \citep{Ji2016}. %Neutron star mergers may have a stronger argument to be the parent site of this level of r-process enhancement in such a small volume. However, magnetorotationally driven supernovae are also a possible solution.
Magnetorotationally driven supernovae may not produce as much r-process material as neutron star mergers (although still more than core-collapse supernovae), they are thought to occur more frequently than neutron-star mergers \todo{[citation needed]}, and the predicted abundance patterns make them observationally indistinguishable from neutron star mergers \citep{Ji2016}.
	
	Europium ($Z=63$), is entirely produced through the r-process. This makes it a very useful signature for identifying stars that show evidence of excess pollution by r-process material. %\todo{CLUMSY, lose the first half-sentence:The recent discovery of the UFD Reticulum II used Europium as evidence of its r-process enhancement, and found the particularly interesting result that the UFD's level of r-process enrichment was too high to have occurred through a core-collapse supernovae event.}
	Historically, UFD galaxies have demonstrated some of the lowest abundance ratios of r-process elements seen in the Milky Way or its satellite systems. However, the analysis of several of Reticulum II's stars show they are greatly enriched, 2-3 orders of magnitude higher than stars found in any other UFD galaxy \citep{Ji2016}. This degree of enrichment supports the theory that a single rare event must have occurred early on. This is supported by the observed heavy element yields which are found to be some 1000 times higher than what is achievable on average from core-collapse supernovae ejecta. It is highly improbable that 1000 supernovae could have contributed to Reticulum II, suggesting that the r-process material must be produced by a rarer astrophysical event such as a binary neutron star merger.
	
	While neutron star mergers or magnetorotationally driven supernovae are likely to produce significant amounts of r-process material, it does not suggest that all r-process material stems from these sources either. Rather, the rarity of these events possibly indicates the rarity of stars found to be heavily enhanced in these elements. One could ask if neutron-star mergers could be responsible for all r-process stars known in the Milky Way? On the other hand, if we assume very little gas mixing, could could core-collapse supernovae create enough material to explain a single r-process star? If neither of these prove to be true then we can confirm that there must be multiple sties to the r-process, providing a step forward in understanding the relative fraction of these mechanisms in the Milky Way.
	
%	 stars we've found? And if neither of these prove to be the case, can we then conclude that there must be multiple sites of r-process element creation throughout the Milky Way, that create these elements to different extents?
	
	To answer these questions it is necessary to identify a large sample of r-process stars.
	 However, r-process stars are very rare, which makes finding this stars difficult. For this reason we must look to massive data sets, such as the LAMOST sky survey. In this letter we report the  discovery of 47 r-process enhanced stars identified from LAMOST.
	In Section 2 we outline the methods used to identify r-process enhanced candidates, and our analysis methods. In Section 3 we discuss what our results imply about the sites of r-process. We provide concluding remarks in Section 4.
	
	
	\begin{figure}
		\includegraphics[width=\columnwidth]{423451}
		\caption{The spectral region around two Europium absorption lines at 4129 and 4205 Angstroms. This particular star's spectra (black) shows clear absorption at these wavelengths, as compared to the data-driven model (red).}
		\label{fig:starindex_423451}
	\end{figure}
	
	\begin{figure}
		\includegraphics[width=\columnwidth]{metalhistpython}
		\caption{The metallicities of the 47 candidate stars.}
		\label{fig:metallicity}
	\end{figure}
	
	\section{Methods}
	
	\subsection{Observations and Candidate Selection}
	
	The LAMOST survey is a low resolution ($R\approx1800$) optical survey that took spectra of stars between 3650-9000 \AA. The second data release from the LAMOST telescope catalogued $\approx$ 2.2 million spectra (low resolution) of stars in the northern sky. Of these 2.2 million, 454,180 giants were fit against a normalised spectra from a sample of 9952 giants \citep{AnnaHo2017}, and from this it became possible to analyse which giants showed an abundance or depletion of particular heavy elements. A machine learning code \textit{The Cannon} was used to fit a predictive model using 9952 LAMOST giant stars that were also found in the higher resolution ($R\approx22500$) APOGEE survey. The labels \textit{The Cannon} predicted were $T_{\rm{eff}}$, $\rm{log_{10}[g]}$, $\rm{[Fe/H]}$, and $\rm{[\alpha/Fe]}$. Normalised spectra was then fit to 454,180 giants from the LAMOST survey and was restricted to between 3905\,\AA\ and 9000\,\AA. For reference, typical uncertainties for these labels are $70\rm{K}$ for $T_{\rm{eff}}$, $0.1$ in $\rm{log_{10}[g]}$, 0.1 in $\rm{[Fe/H]}$, and 0.04 in $\rm{[\alpha/H]}$.
	
	We selected potential candidate stars with an over-abundance of Europium by searching for deviations from the normalized spectra supplied by \textit{The Cannon}. A negative deviation implies an enhancement in a particular element while a positive deviation implies a depletion (compared to what the data-driven model predicted). 
	
	At two absorption lines for Europium (4129\,\AA\ and 4205\,\AA) we fitted Gaussian profiles to each star in the 454,180 giants from LAMOST DR2, and we recorded the amplitude and the wavelength of each, as well as their corresponding errors. Candidate Europium enhanced stars were selected by identifying stars that met the following criteria:
	
	\begin{itemize}
		\item The amplitude of their deviation from the model needed to be less than $\rm{A}<-0.05$;
		\item The signal to noise (S/N) must be greater than 30;
		\item The deviation from the required wavelength in question must be less than 2 \AA;
		\item The $\chi_{\rm{r}}^{2}$ value (where the $\chi_{\rm{r}}^{2}$ value represents how well the model fit the spectrum) must be less than 3, $\chi_{\rm{r}}^{2}<3$.
		\item All of these conditions must be met for both 4129 \AA\ and 4205 \AA\ for us to consider them sufficiently Eu enhanced.
	\end{itemize}   
	
	We gathered an initial sample of 62 candidates out of the 454180 LAMOST giant sample. These were then visually scrutinised to identify those that lacked visual evidence that they were enhanced. After this process we were left with 47 potentially r-process enhanced candidates, 22 of which stand out as very strong candidates. An example of one of our candidates is shown in Figure \ref{fig:starindex_423451}.
	
	\section{The r-process}
	For the r-process to occur there must be a very high neutron density, such that neutron capture must be able to occur before the nucleus undergoes a $\beta$-decay (assuming the nucleus is in an unstable state). The conditions that can create this are rare. Events such as supernovae or neutron star mergers list among the most favoured sites.
	
	\subsection{Selection effects}
	The LAMOST giants we analyse belong to a broad range of metallicity ($\rm{[Fe/H]} -1.6\ \rm{to}\ 0.6$) as seen in Figure \ref{fig:metallicity}, and don't cluster towards any position in the observed sky. These characteristics argue that the abundance we see could not have originated from a common era or site. 
	
	The previous sample of r-process stars, before the release of this letter, almost exclusively consists of metal-poor stars. This fact is not to be confused with saying that most r-process stars found should be metal-poor stars, rather that at low metallicities we know that the s-process will not contribute, so neutron capture elements formed from these stars would be only from the r-process. Over half the current known r-process stars come from a very metal poor ultra faint dwarf galaxy, Reticulum II. However, Europium we know to be totally made from the r-process, therefore even at metallicities where the s-process contributes, Europium is a valuable indicator of r-process synthesis.
	
	\subsection{Multiple r-process sites?}
	While it is possible that all currently known r-process enhancements can be made by neutron star mergers, we still observe stellar populations in the galaxy which show particularly weak enrichment from r-process elements ([Eu/Fe] < 0.7). These enhancements imply that an additional site of r-process element synthesis is likely to be active. As we show in \ref{rates}  neutron star merger rates may show that all heavily enriched stars can be enhanced by these phenomena within an adequate time-scale. However, we are not able to assume that as a consequence of the merger some stars are greatly enhanced, while others are only very slightly enhanced. It is more accurate to propose that at least two sites exist, with one (potentially mergers) producing a massive amount of enhanced material, and another only producing a small amount, such as supernovae. 
	
	\subsection{Core-collapse supernovae (CCSN)}
	Core-collapse supernova are one of the sites initially proposed to have a high enough neutron density to create elements synthesised by the r-process \citep{Burbidge1957}. 
	
	As of yet our knowledge is incomplete on the exact mechanism that allows material with a high enough neutron density to escape the supernovae event.
	One possible mechanism is the action of high-entropy neutrino winds being released from a proto-neutron star during core-collapse. In this stage of the collapse, this neutrino wind is driving mass loss and could be instrumental in the expulsion of the r-process material in supernovae conditions. However, how much enhanced material is released from the event is very dependant on the amount of material that is allowed to fall back onto the remnant at the arrival of the reverse shock, and so often the supernovae may not release a considerable amount of material \citep{Woosley1992, Burrows1995}. 
	
	Regardless of the exact mechanism, core-collapse supernovae seem to release $M_{\rm{Eu}}\approx10^{-7.5} M_{\odot}$ per event \citep{Argast2004} . While this may not be a particularly significant amount, core-collapse supernovae occur frequently $44700\ \rm{Gpc}^{-3} \rm{yr^{-1}}$ \citep{Li2011}, and so it isn't unreasonable to suggest that, with very little mixing occurring, that much if not all r-process material could have been formed by these and recycled in the stars we present in this letter.
	
	\subsection{Neutron star mergers}
	\label{NSmerg}
	Neutron star mergers are another possible site of r-process element nucleosynthesis \citep{Kasen2017,Hotok2013,Drout2017}. They satisfy the conditions necessary to create a high neutron density by means of interacting tidal forces between the stars, neutron dense material is accreted off the stars, allowing r-process element synthesis to occur. This specific process produces considerably more enriched material than core-collapse supernovae ($M_{\rm{Eu}}\approx10^{-4.5} M_{\odot}$) \citep{Goriely2011}. However, the direct interaction and merger of a binary neutron star pair occurs very infrequently, with currently only one direct observation, and with a frequency estimate of $1000\ \rm{Gpc^{-3}yr^{-1}}$ \citep{LIGO2016}. The significant amount of material that these mergers release can compensate for their low frequency and so, synonymous with core-collapse supernovae, it is not unreasonable to suggest that they are responsible for most if not all known r-process material. 

	The recent discovery of the UFD Reticulum II \citep{Ji2016} is evidence against the case of a single r-process site. Of nine stars observed in Reticulum II, 7 of them were shown to be highly enriched in r-process elements ($\rm{[Eu/Fe]\approx1.7}$) which suggests a single prolific event having occurred in close proximity to the galaxy early in it's formation, such as a neutron star merger. However, two of the stars observed were only weakly enhanced in r-process elements which suggests that there must have been a second, less efficient, form of r-process enhancement present.
	
	\subsection{The rates of r-process events and what they imply}
	\label{rates}
	From the recent advanced LIGO data \citep{LIGO2016}, we are able to find an estimate for a plausible frequency at which neutron star mergers occur, as mentioned in section \ref{NSmerg}. As well as a reliable estimate of core-collapse supernovae rates from \citet{Li2011}.
	
	From these two rates, we can for the first time to our knowledge, estimate the amount of r-process stars that should be present in the Milky Way, enhanced both by purely neutron star mergers and purely core-collapse supernovae. We do this assuming instant recycling, and that no r-process matter pollutes stars by any other means than these.
	
	From these rates we can find an estimate of how many r-process enhanced giants should be present from an approximate number of the Milky Way's total stars. As well as an estimate of how many enhanced stars, giants or otherwise, that should be present in the Milky Way.
	
	In preparing these calculations we assumed a constant star formation rate, 2 M$_\odot yr^{-1}$, in the disk of the Milky Way, $\rm{r=15kpc}$, with a typical IMF and giant star birth rate of approximately $0.3\ $ stars per year. We also assumed the lifetime between $\log{g} = 3.2$ and the end of the core-helium burning phase is about $250\,{\rm Myr}$. Therefore if we assume a steady-state system, the number of giant stars in the Milky Way comes to approximately $7.53\times10^7$. This allows us to find the percentage of the Milky Ways stars that our sample represents.
	
	We also state that a star is said to be enhanced if [Eu/Fe] $\geq$ 0.7, and performed the lower-bound estimate that all of our 47 stars exhibit this abundance. The amount of r-process material expelled by neutron star mergers is the same as previously discussed, using Europium as our identifier, of $M_{Eu}\approx10^{-4.5} M_{\odot}$.
	
    It was found that in the Milky Way there should be present 9,896 r-process enhanced stars from solely neutron star mergers. Taking into account that this should be considered a lower-bound estimate (by assuming a lower bound Europium abundance) this is a startlingly large difference from the known 60 r-process stars in the Milky Way.
    
    Scaling down from the 250 billion stars in the Milky Way to our sample size of $\approx450000$ giants, the amount of enhanced stars we expect to find comes to less than one (0.0178). If we are to extend our sample from the 450000 giants, and instead use the entire LAMOST data release 2 (a survey of $\approx2.2$ million stars), we predict that we should still have found less than a single star (0.087) enhanced in r-process elements. While again this should be considered a lower bound, it is nonetheless a startling difference from the 47 that were found.
    
    We can do a similar calculation with our rate of core-collapse supernovae. Taking the same assumptions with minimal mixing, as well as both the ejecta ($M_{Eu}\approx10^{-4.5} M_{\odot}$) and rate ($44700\ \rm{Gpc}^{-3} \rm{yr^{-1}}$) previously discussed, we find that in the Milky Way there should be present 442 r-process enhanced stars made solely from core-collapse supernovae.
    
    Were we to scale this down to our sample size of $\approx450000$ giants we arrive at a figure of $8.0\times10^{-4}$ stars that should have been observed. Scaling again to the full LAMOST sample we would then expect to find $3.9\times10^{-3}$ enhanced stars. 
    
    From these estimates we can find an approximation for whether or not either neutron star mergers or core-collapse supernovae by themselves are sufficient to enrich our entire r-process enhanced catalogue. 
    
    We find that, by using the assumptions listed, neutron star mergers could create enough r-process material to enrich above the point of [Eu/Fe] > 0.7 for all 60 known r-process enhanced stars within $13.87\times10^9$ years, very close to the age of the universe. From our rates of core-collapse supernovae however, we find that the earliest time-scale at which they can create this level of r-process material is $3.1\times10^{11}$ years, not at all within the age of the universe. This illustrates, with our current knowledge of core-collapse supernovae frequency, and the amount of r-process material they eject, core-collapse supernovae cannot be the sole source of r-process enrichment in the Milky Way.
    
    This is reinforced by our r-process enhanced stars found in the LAMOST sample. Our 47 stars represent greater then 10\% of all enhanced stars present in the Milky Way if core-collapse supernovae is considered to be the singular source of r-process enhanced stars. However, the LAMOST data release 2 only surveyed 0.00088\% of the Milky Way. It is highly unlikely that 10\% of the entire r-process enhanced population of the Milky Way exists within our stellar sample, it is then concluded that core-collapse supernovae can not be the only source of r-process material. 
    
    \todo{For Andy (haven't written this bit nicely this was a thought dump): Let us consider that both of these estimates are correct, and the rates they are based on have been roughly accurate over the age of the universe (although they should have increased with time, which is actually a bit worse for us). Then we are left with our previous prediction of 9,896 stars enhanced purely by neutron star mergers, and another 442 stars enhanced purely by core-collapse supernovae. This leaves us with a new sample of r-process enhanced stars that should exist in the Milky Way of 10,338. 
    At a rough estimate, the LAMOST data release 2 took spectra for 0.00088\% of the Milky Way's stars (assuming an average amount of stars in the Milky Way of 250 billion). In this very small percentage, 47 r-process enhanced stars were found. Then, if we are able to assume that there is nothing particularly special about the area of the sky that LAMOST surveyed, scaling this size to that of the Milky Way we should find 5,340,892 r-process enhanced stars. Clearly there is something wrong.}
    
	
	\section{Conclusions}
	We present the largest to date sample of r-process element enhanced stars to date. Our sample is found at a wide range of metallicities, contrary to previous discoveries that focussed on metal-poor r-process enhanced stars. Similarly they are not clumped in either temperatures, $log(g)$, or radial velocity. They don't cluster to any one end of the H-R diagram, highlighting no correlation between type of star, they are observed both in the disk and halo, and are not clustered anywhere in the sky. In addition to this we also present the first ever estimate to our knowledge, from information provided by advanced LIGO, that neutron star mergers possess the ability to synthesise all currently known r-process material in a feasible time-scale. We also provide evidence that we can confidently conclude core-collapse supernovae can not possibly be the only progenitors of r-process material in the Milky Way Galaxy.
	
	We recommend follow-up high resolution spectra be obtained for each of the 47 stars to ascertain how enhanced they are. To find the origin of these stars and to show whether or not they may come from a similar progenitor site it is necessary to find orbits for them.
	
	
	% The best way to enter references is to use BibTeX:
	
	\bibliographystyle{mnras}
	%\bibliography{paper} % if your bibtex file is called example.bib
	\begin{thebibliography}{99}
		\bibitem[\protect\citeauthoryear{Abbott, B.P. et al.}{2016}]{LIGO2016}
		Abbot B.P. et al. 2016, Living Reviews in Relativity, 19, 1
		\bibitem[\protect\citeauthoryear{Argast, D. et al.}{2004}]{Argast2004}
		Argast D., Samland M., Thielmann F-K. \& Qian Y-Z. 2004, Astronomy and Astrophysics, 416
		\bibitem[\protect\citeauthoryear{Burbidge, E. et al.}{1957}]{Burbidge1957}
		Burbidge E. M., Burbidge G. R., Fowler W. A., \& Hoyle F. 1957, Review of Modern Physics, 29, 4
		\bibitem[\protect\citeauthoryear{Burrows, A. et al.}{1995}]{Burrows1995}
		Burrows, A., Hayes, J., \& Fryxell, Bruce A. 1992, ApJ, 450, 830
		\bibitem[\protect\citeauthoryear{Drout, M. R. et al.}{2017}]{Drout2017}
		Drout, M. R. et al. 2017, Science
		\bibitem[\protect\citeauthoryear{Goriely, S. et al.}{2011}]{Goriely2011}
		Goriely S., Bauswein A. \& Janka H-T. 2011, , 738, L32
		\bibitem[\protect\citeauthoryear{Ho et al.}{2017}]{AnnaHo2017}
		Ho A. Y. Q., Ness M. K., Hogg David W., Rix H-W., Liu C., Yang F., Zhang Y., Hou Y.,\& Wang Y. 2017, ApJ, 836, 1
		\bibitem[\protect\citeauthoryear{Hotokezaka, K. et al.}{2013}]{Hotok2013}
		Hotokezaka, K., Kiuchi, K., Kyutoku, K., Okawa, H., Skiguchi, Y., Shibata, M., \& Taniguchi, K. 2013, APS
		\bibitem[\protect\citeauthoryear{Ji, A. P. et al.}{2016}]{Ji2016}
		Ji A. P., Frebel A., Chiti A., \& Simon J. D. 2016, Nature, 531
		\bibitem[\protect\citeauthoryear{Kasen, D. et al.}{2017}]{Kasen2017}
		Kasen, D., Metzger, B., Barnes, J., Quataert, E., \& Ramirez-Ruiz, E. 2017, Nature, 551, 7678
		\bibitem[\protect\citeauthoryear{Li, W. et al.}{2011}]{Li2011}
		Li, W., Chornock, R., Leaman, J., Filippenko, A. V., Poznanski, D., Wang, X., Ganshalingam, M., \& Mannucci, F. 2011, Royal Astronomical Society, 412, 1473
		\bibitem[\protect\citeauthoryear{Sneden, C. et al.}{2008}]{Sneden2008}
		Sneden C., Cowan J.J., \& Gallino R. 2013, Annual review of Astronomy \& Astrophysics, 46, 1
		\bibitem[\protect\citeauthoryear{Woosley, S.E. et al.}{1992}]{Woosley1992}
		Woosley S.E, \& Hoffman R.D. 1992, ApJ, 395, 1
	\end{thebibliography}
	
	% Please add the following required packages to your document preamble:
	% \usepackage{booktabs}
	% Please add the following required packages to your document preamble:
	% \usepackage{booktabs}
	\begin{table*}
		\centering
		\label{Data for the 47}
		\begin{tabular}{@{}cccccccccccc@{}}
			\toprule
			\textbf{2MASS ID}   & \textbf{RA} & \textbf{DEC} & \textbf{S/N}        	& \textbf{$\boldsymbol{V_{\rm{r}}}$}   & \textbf{$\boldsymbol{T_{\rm{eff}}}$} & \textbf{Log(g)}	& \textbf{$\boldsymbol{[\rm{Fe/H}]}$} & \textbf{$\boldsymbol{[\alpha/\rm{H}]}$} & \textbf{$\boldsymbol{\chi_{\rm{r}}^{2}}$} & \textbf{{[}Eu/Fe{]}} & \textbf{Error} \\
			-               	& {[}H:M:S{]}   & {[}H:M:S{]}	& {[}$\rm{pixel}^{-1}${]} & {[}$\rm{km\ s}^{-1}${]} & {[}K{]}             	& {[}$\rm{cm\ s}^{-2}${]} & {[}dex{]}          	& {[}dex{]}              	& -                        	& {[}dex{]}        	& {[}dex{]}  	\\ \midrule
			J130303.65+260837.0 & 13:03:03.65 & +26:08:37.0  & 36.48               	& -37.77              	& 4851.24             	& 2.98           	& -0.54              	& 0.19                   	& 0.42                     	& 1.39             	& 0.31       	\\
			J062733.76+280814.1 & 06:27:33.77 & +28:08:14.2  & 39.08               	& 38.67               	& 4502.58             	& 1.92           	& -0.15              	& 0.00                   	& 0.35                     	& 1.43             	& 0.35       	\\
			J070035.19+103443.5 & 07:00:35.19 & +10:34:43.5  & 41.77               	& 17.69               	& 4752.08             	& 2.44           	& -0.43              	& 0.06                   	& 0.33                     	& 1.09             	& 0.63       	\\
			J070107.18+125948.2 & 07:01:07.19 & +12:59:48.2  & 33.72               	& 46.17               	& 4452.61             	& 2.51           	& 0.47               	& 0.02                   	& 0.36                     	& 1.30             	& 0.31       	\\
			J063400.95+420904.3 & 06:34:00.96 & +42:09:04.4  & 34.09               	& 9.89                	& 4315.56             	& 1.62           	& -0.50              	& 0.10                   	& 0.43                     	& 1.19             	& 3.10       	\\
			J220118.66+033558.6 & 22:01:18.66 & +03:35:58.6  & 35.11               	& -108.23             	& 4790.06             	& 2.51           	& -1.36              	& 0.33                   	& 0.37                     	& 1.44             	& 0.26       	\\
			J221220.27+032735.4 & 22:12:20.27 & +03:27:35.5  & 35.16               	& -12.89              	& 4867.34             	& 3.12           	& 0.11               	& 0.14                   	& 0.35                     	& 1.52             	& 0.20       	\\
			J060021.29+384840.6 & 06:00:21.29 & +38:48:40.6  & 30.03               	& 32.68               	& 4305.08             	& 2.10           	& 0.15               	& -0.01                  	& 0.54                     	& 1.49             	& 0.35       	\\
			J081632.10-053703.4 & 08:16:32.11 & -05:37:03.4  & 59.16               	& 44.67               	& 4541.14             	& 2.45           	& 0.57               	& 0.03                   	& 0.75                     	& 1.34             	& 0.26       	\\
			J064641.14+235318.2 & 06:46:41.14 & +23:53:18.3  & 36.77               	& 25.78               	& 4325.56             	& 1.78           	& -0.10              	& 0.02                   	& 0.42                     	& 1.41             	& 0.27       	\\
			J025225.88+324308.1 & 02:52:25.88 & +32:43:08.2  & 34.79               	& 24.88               	& 4797.58             	& 2.52           	& -0.23              	& 0.05                   	& 0.24                     	& 1.31             	& 0.29       	\\
			J061822.02+015126.7 & 06:18:22.02 & +01:51:26.8  & 36.75               	& 45.27               	& 4617.95             	& 2.43           	& -0.12              	& 0.04                   	& 0.34                     	& 1.42             	& 0.30       	\\
			J083049.21+304144.3 & 08:30:49.21 & +30:41:44.3  & 39.5                	& -125.31             	& 4859.92             	& 2.28           	& -1.63              	& 0.33                   	& 0.41                     	& 1.52             	& 0.20       	\\
			J035949.12+302104.4 & 03:59:49.13 & +30:21:04.4  & 30.42               	& 26.98               	& 4164.41             	& 1.79           	& -0.12              	& 0.00                   	& 0.36                     	& 1.47             	& 0.32       	\\
			J074852.24+075016.5 & 07:48:52.25 & +07:50:16.6  & 33.47               	& 36.87               	& 4405.84             	& 1.94           	& -0.48              	& 0.07                   	& 0.30                     	& 1.44             	& 0.30       	\\
			J163116.81+002349.5 & 16:31:16.81 & +00:23:49.5  & 32.63               	& -35.98              	& 4628.92             	& 2.53           	& 0.01               	& 0.18                   	& 0.29                     	& 1.44             	& 0.32       	\\
			J073532.54+215850.0 & 07:35:32.55 & +21:58:50.0  & 35.61               	& 35.98               	& 4859.66             	& 3.27           	& 0.19               	& 0.06                   	& 0.44                     	& 1.19             	& 0.79       	\\
			J164023.01+164552.5 & 16:40:23.02 & +16:45:52.6  & 44.59               	& -34.18              	& 4964.91             	& 2.96           	& 0.36               	& 0.07                   	& 0.35                     	& 1.27             	& 0.32       	\\
			J153333.12+510621.8 & 15:33:33.13 & +51:06:21.9  & 49.45               	& -17.99              	& 4861.78             	& 2.51           	& -0.49              	& 0.22                   	& 0.46                     	& 1.31             	& 0.28       	\\
			J175737.71+521523.1 & 17:57:37.71 & +52:15:23.2  & 38.52               	& -40.47              	& 4861.38             	& 3.36           	& -0.34              	& 0.10                   	& 0.55                     	& 1.51             	& 0.24       	\\
			J184125.86+432807.8 & 18:41:25.86 & +43:28:07.9  & 46.94               	& -93.84              	& 4703.92             	& 2.61           	& -0.26              	& 0.27                   	& 0.33                     	& 1.20             	& 0.34       	\\
			J195234.41+470859.4 & 19:52:34.41 & +47:08:59.5  & 68.57               	& -24.28              	& 4596.44             	& 2.54           	& 0.55               	& 0.07                   	& 0.82                     	& 0.95             	& 0.48       	\\
			J215146.68+304016.0 & 21:51:46.68 & +30:40:16.0  & 73.25               	& -28.78              	& 4566.21             	& 2.48           	& 0.64               	& 0.04                   	& 0.71                     	& 0.88             	& 0.45       	\\
			J075116.55+220137.8 & 07:51:16.56 & +22:01:37.9  & 32.87               	& 90.24               	& 4810.60             	& 2.70           	& -0.33              	& 0.02                   	& 0.30                     	& 1.35             	& 0.31       	\\
			J003929.43+430039.5 & 00:39:29.43 & +43:00:39.6  & 33.14               	& -104.33             	& 4190.26             	& 1.87           	& -0.01              	& 0.07                   	& 0.41                     	& 1.35             	& 0.33       	\\
			J040122.07+461504.5 & 04:01:22.08 & +46:15:04.5  & 35.43               	& -95.03              	& 4305.88             	& 1.75           	& 0.02               	& 0.10                   	& 0.57                     	& 1.37             	& 0.35       	\\
			J040815.45+464920.1 & 04:08:15.46 & +46:49:20.1  & 33.05               	& 16.19               	& 4855.16             	& 2.65           	& -0.11              	& 0.00                   	& 0.30                     	& 1.48             	& 0.30       	\\
			J071206.83+213038.0 & 07:12:06.84 & +21:30:38.1  & 32.66               	& 29.08               	& 4432.03             	& 1.75           	& -0.38              	& 0.11                   	& 0.40                     	& 1.47             	& 0.28       	\\
			J222820.22+340750.3 & 22:28:20.22 & +34:07:50.3  & 34.52               	& -45.27              	& 4419.96             	& 2.06           	& -0.07              	& 0.11                   	& 0.49                     	& 1.37             	& 0.31       	\\
			J222310.49+344357.4 & 22:23:10.50 & +34:43:57.5  & 54.06               	& -20.09              	& 4232.66             	& 1.87           	& -0.01              	& 0.03                   	& 0.86                     	& 1.22             	& 0.31       	\\
			J222101.61+365923.6 & 22:21:01.62 & +36:59:23.7  & 30.81               	& -27.88              	& 4360.28             	& 2.23           	& 0.20               	& 0.04                   	& 0.45                     	& 1.47             	& 0.29       	\\
			J052327.44+571126.8 & 05:23:27.45 & +57:11:26.8  & 33.54               	& -24.88              	& 4806.24             	& 3.08           	& 0.24               	& 0.07                   	& 0.45                     	& 1.43             	& 0.25       	\\
			J235212.27+492548.4 & 23:52:12.27 & +49:25:48.4  & 39.93               	& -95.03              	& 5047.36             	& 3.30           	& -0.43              	& 0.22                   	& 0.29                     	& 1.28             	& 0.26       	\\
			J074102.63+130520.3 & 07:41:02.64 & +13:05:20.4  & 40.83               	& 49.17               	& 5000.84             	& 3.00           	& -0.38              	& -0.02                  	& 0.30                     	& 1.35             	& 0.27       	\\
			J061559.10+470344.9 & 06:15:59.11 & +47:03:44.9  & 36.84               	& -18.29              	& 4811.98             	& 2.82           	& -0.59              	& 0.07                   	& 0.44                     	& 1.00             	& 1.33       	\\
			J054704.12+600827.1 & 05:47:04.13 & +60:08:27.1  & 33.19               	& -23.33              	& 4770.80             	& 2.64           	& -0.10              	& 0.06                   	& 0.40                     	& 1.46             	& 0.26       	\\
			J055548.33+611537.1 & 05:55:48.33 & +61:15:37.1  & 33.6                	& -40.88              	& 4850.39             	& 2.57           	& -0.46              	& 0.13                   	& 0.22                     	& 1.42             	& 0.26       	\\
			J042723.73+271546.8 & 04:27:23.73 & +27:15:46.9  & 45.73               	& -40.17              	& 4211.91             	& 1.72           	& -0.04              	& 0.05                   	& 0.79                     	& 1.33             	& 0.34       	\\
			J035702.05+281335.2 & 03:57:02.06 & +28:13:35.3  & 31.82               	& -44.67              	& 4712.89             	& 2.44           	& 0.01               	& 0.16                   	& 0.46                     	& 1.28             	& 0.38       	\\
			J031030.13+531759.9 & 03:10:30.13 & +53:17:59.9  & 35.34               	& -16.79              	& 4699.75             	& 2.34           	& 0.21               	& 0.00                   	& 0.89                     	& 1.51             	& 0.21       	\\
			J091256.32+353008.3 & 09:12:56.32 & +35:30:08.3  & 73.39               	& 20.39               	& 4441.95             	& 2.33           	& 0.41               	& 0.02                   	& 0.76                     	& 0.92             	& 0.39       	\\
			J071534.00+211353.8 & 07:15:34.00 & +21:13:53.9  & 31.68               	& 42.27               	& 4892.00             	& 2.57           	& -0.72              	& 0.11                   	& 0.34                     	& 0.59             	& 0.29       	\\
			J101317.14+122928.8 & 10:13:17.14 & +12:29:28.8  & 45.54               	& 300.69              	& 5055.95             	& 2.24           	& -1.34              	& 0.33                   	& 0.38                     	& 1.46             	& 0.24       	\\
			J161326.00+073218.1 & 16:13:26.00 & +07:32:18.2  & 48.45               	& -50.07              	& 5086.79             	& 3.40           	& -0.68              	& 0.29                   	& 0.93                     	& 1.38             	& 0.24       	\\
			J173711.85+101126.6 & 17:37:11.85 & +10:11:26.7  & 30.01               	& -33.58              	& 4991.07             	& 3.49           	& -0.05              	& 0.17                   	& 0.41                     	& 1.45             	& 0.28       	\\
			J173041.92+125640.5 & 17:30:41.92 & +12:56:40.5  & 52.67               	& 20.39               	& 4986.98             	& 3.42           	& -0.45              	& 0.22                   	& 0.30                     	& 1.09             	& 0.43       	\\
			J172929.42+120133.3 & 17:29:29.43 & +12:01:33.3  & 39.91               	& 14.39               	& 4597.08             	& 2.67           	& 0.24               	& 0.05                   	& 0.67                     	& 1.02             	& 0.80       	\\ \bottomrule
		\end{tabular}
		\caption{Abundances and additional information of the 47 enhanced stars.}
	\end{table*}
	
	
	
	
	
	% Don't change these lines
	%\bsp	% typesetting comment
	\label{lastpage}
\end{document}
